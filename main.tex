\section*{Treść zadania}
\begin{enumerate}
    \item Zaprojektować licznik rewersyjny liczący w przód od 0..9 i w tył od 9..0 w kodzie (UZUPEŁNIJ). Licznik powinien zawierać:
    \begin{itemize}
        \item COUNTING DIRECTION (CD). Wartość logiczna 0 oznacza liczenie w przód.
        \item SET VALUE (SV). Sygnał nadrzędny asynchroniczny, który ustawia aktywne zero wartość licznika.
        \item CLOCK ENABLE (CE). Licznik liczy przy CE $ = 1$ i nie liczy przy CE $= 0$.
    \end{itemize}
    \item Przeprowadzić symulację układu, w której w formie sekwencyjnej pojawiają się zmiany sygnałów: \\
    Zegarowego, CE, CD, SV. Sprawdzić wszystkie funkcje licznika.
    
\end{enumerate}