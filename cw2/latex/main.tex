\tableofcontents
\clearpage

\section{Treść zadania}

Utworzyć moduł z czterema sygnałami wejściowymi: 8-bitową liczbą $A$, 8-bitową liczbą $B$, dwubitowym wyboru operacji oraz bitem $C$ (przeniesienie/pożyczka) oraz z dwoma sygnałami wyjściowymi: 8-bitową liczbą wyniku oraz 4-bitowym słowem statusowym w, którym mają się znaleźć bity: $C$- (przeniesienie/pożyczka), bit parzystości typu EVEN wyniku, bit $Z$ - znacznik wartości zero, bit-OV overflow określający przekroczenie zakresu dla arytmetyki liczb ze znakiem.

\vspace{0.5cm}

Zaprojektować cztery podukłady, dwa w formie TASK-u , mają one stanowić iteracyjny blok dla poszczególnych zadań, a dwa w formie wyrażeń logicznych:

\begin{enumerate}
    \item arytmetyczna suma z przeniesieniem (wejścia: przeniesienie, bit liczby $A$, bit liczby $B$, wyjścia: bit sumy, bit przeniesienia)
    \item arytmetyczna różnica ($A-B$) z pożyczką (wejścia: pożyczka, bit liczby A, bit liczby B, wyjścia: bit sumy, bit pożyczki)
    \item logiczna suma dwóch liczb 8-bitowych
    \item logiczny iloczyn dwóch liczb 8-bitowych
\end{enumerate}

\vspace{0.5cm}

Następnie zaprojektować układ  wykorzystując utworzone wcześniej podukłady.  

\vspace{0.5cm}

Układ w zależności od sygnału wyboru 1=0, ma realizować zadania arytmetyczne (wybor0=zero) sumę liczb z  bitem C jako przeniesienie lub (wybor 0 = jeden) różnicę liczb (A-B) z bitem $C$ jako pożyczka. Uzupełnić układ elementami kombinacyjnym wyznaczającymi sygnały $Z$ i OV oraz bit parzystości typu EVEN.

\vspace{0.5cm}

Układ dla sygnału wyboru 1 = 1, ma realizować zadania logiczne (wybor0=zero) sumę logiczną wyrażeń, (wybor0=jeden) iloczyn logiczny wyrażeń. Uzupełnić układ elementami kombinacyjnym wyznaczającymi sygnały $Z$ oraz bit parzystości typu EVEN. Pozostałe dwa bity wynikowego słowa statusowego należy ustawić na zero.
\clearpage

\section{Arytmetyczna suma i różnica dwóch liczb}
Do działań arytmetycznych stworzyliśmy osobny ''task'', przy pomocy którego obliczane jest równanie, z elementem przeniesienia/pożyczenia

\begin{lstlisting}
task sumator(input la, lb, p, output wynik, prze);
	begin
		prze = (la & lb) | (la & p) | (p & lb);
		wynik = (la ^ lb) ^ p;
	end
endtask

task roznica(input la, lb, p, output wynik, prze);
	begin
		prze = (~la & lb) | (lb & p) | (~la & p);
		wynik = (~la & ~lb & p) | (~la & lb & ~p) | (la & ~lb & ~p) | (la & lb & p);
	end
endtask
\end{lstlisting}

\section{Logiczna suma i iloczyn dwóch liczb}
Z powodu tego jak funkcjonuje oprogramowanie, wystarczyła jednolinijkowa definicja logiki dla obydwu przypadków. Została ona umieszczona w funkcji wyboru, przełączanej później switch'ami.

\begin{lstlisting}
	else if(wybor == 2'b10)
	begin
		wynik = liczbaA | liczbaB;
	end
	
	else if(wybor == 2'b11)
	begin
		wynik = liczbaA & liczbaB;
	end
\end{lstlisting}
\clearpage

\section{Całość kodu}

\input{verilog_main.tex}

\section{Przypisanie przycisków}

\input{verilog_przypisania.tex}