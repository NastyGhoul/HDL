\tableofcontents
\clearpage

\section{Treść zadania}

Utworzyć moduł z czterema sygnałami wejściowymi: 8-bitową liczbą $A$, 8-bitową liczbą $B$, dwubitowym wyboru operacji oraz bitem $C$ (przeniesienie/pożyczka) oraz z dwoma sygnałami wyjściowymi: 8-bitową liczbą wyniku oraz 4-bitowym słowem statusowym w, którym mają się znaleźć bity: $C$- (przeniesienie/pożyczka), bit parzystości typu EVEN wyniku, bit $Z$ - znacznik wartości zero, bit-OV overflow określający przekroczenie zakresu dla arytmetyki liczb ze znakiem.

\vspace{0.5cm}

Zaprojektować cztery podukłady, dwa w formie TASK-u , mają one stanowić iteracyjny blok dla poszczególnych zadań, a dwa w formie wyrażeń logicznych:

\begin{enumerate}
    \item arytmetyczna suma z przeniesieniem (wejścia: przeniesienie, bit liczby $A$, bit liczby $B$, wyjścia: bit sumy, bit przeniesienia)
    \item arytmetyczna różnica ($A-B$) z pożyczką (wejścia: pożyczka, bit liczby A, bit liczby B, wyjścia: bit sumy, bit pożyczki)
    \item logiczna suma dwóch liczb 8-bitowych
    \item logiczny iloczyn dwóch liczb 8-bitowych
\end{enumerate}

\vspace{0.5cm}

Następnie zaprojektować układ  wykorzystując utworzone wcześniej podukłady.  

\vspace{0.5cm}

Układ w zależności od sygnału wyboru 1=0, ma realizować zadania arytmetyczne (wybor0=zero) sumę liczb z  bitem C jako przeniesienie lub (wybor 0 = jeden) różnicę liczb (A-B) z bitem $C$ jako pożyczka. Uzupełnić układ elementami kombinacyjnym wyznaczającymi sygnały $Z$ i OV oraz bit parzystości typu EVEN.

\vspace{0.5cm}

Układ dla sygnału wyboru 1 = 1, ma realizować zadania logiczne (wybor0=zero) sumę logiczną wyrażeń, (wybor0=jeden) iloczyn logiczny wyrażeń. Uzupełnić układ elementami kombinacyjnym wyznaczającymi sygnały $Z$ oraz bit parzystości typu EVEN. Pozostałe dwa bity wynikowego słowa statusowego należy ustawić na zero.
\clearpage

\section{Arytmetyczna suma i różnica dwóch liczb}
Do działań arytmetycznych stworzyliśmy osobny ''task'', przy pomocy którego obliczane jest równanie, z elementem przeniesienia/pożyczenia

\begin{lstlisting}
task sumator(input la, lb, p, output wynik, prze);
	begin
		prze = (la & lb) | (la & p) | (p & lb);
		wynik = (la ^ lb) ^ p;
	end
endtask

task roznica(input la, lb, p, output wynik, prze);
	begin
		prze = (~la & lb) | (lb & p) | (~la & p);
		wynik = (~la & ~lb & p) | (~la & lb & ~p) | (la & ~lb & ~p) | (la & lb & p);
	end
endtask
\end{lstlisting}

\section{Logiczna suma i iloczyn dwóch liczb}
Z powodu tego jak funkcjonuje oprogramowanie, wystarczyła jednolinijkowa definicja logiki dla obydwu przypadków. Została ona umieszczona w funkcji wyboru, przełączanej później switch'ami.

\begin{lstlisting}
	else if(wybor == 2'b10)
	begin
		wynik = liczbaA | liczbaB;
	end
	
	else if(wybor == 2'b11)
	begin
		wynik = liczbaA & liczbaB;
	end
\end{lstlisting}
\clearpage

\section{Całość kodu}

\begin{lstlisting}
`timescale 1ns / 1ps


module main(
    input [7:0] liczbaA,
    input [7:0] liczbaB,
	 input [1:0] wybor,
	 input bitP,
	 output reg[7:0] wynik,
	 output C,
	 output EVEN,
	 output Z,
	 output OV
    );

reg [7:0] pw;
reg z1;
reg z2;
reg zw;
reg ov;

task sumator(input la, lb, p, output wynik, prze);
	begin
		prze = (la & lb) | (la & p) | (p & lb);
		wynik = (la ^ lb) ^ p;
	end
endtask

task roznica(input la, lb, p, output wynik, prze);
	begin
		prze = (~la & lb) | (lb & p) | (~la & p);
		wynik = (~la & ~lb & p) | (~la & lb & ~p) | (la & ~lb & ~p) | (la & lb & p);
	end
endtask

always @(wybor or liczbaB or liczbaA or bitP)
begin
	pw = 0;
	ov = 0;

	if (wybor == 2'b00)
	begin
		z1 = liczbaA[7];
		z2 = liczbaB[7];
		
		sumator(liczbaA[0],liczbaB[0],bitP,wynik[0], pw[0]);
		sumator(liczbaA[1],liczbaB[1],pw[0],wynik[1], pw[1]);
		sumator(liczbaA[2],liczbaB[2],pw[1],wynik[2], pw[2]);
		sumator(liczbaA[3],liczbaB[3],pw[2],wynik[3], pw[3]);
		sumator(liczbaA[4],liczbaB[4],pw[3],wynik[4], pw[4]);
		sumator(liczbaA[5],liczbaB[5],pw[4],wynik[5], pw[5]);
		sumator(liczbaA[6],liczbaB[6],pw[5],wynik[6], pw[6]);
		sumator(liczbaA[7],liczbaB[7],pw[6],wynik[7], pw[7]);
		
		zw = wynik[7];
		ov = (~z1 & ~z2 & zw) | (z1 & z2 & ~zw);
	end
	
	else if(wybor == 2'b01)
	begin
		z1 = liczbaA[7];
		z2 = liczbaB[7];
	
		roznica(liczbaA[0],liczbaB[0],bitP,wynik[0], pw[0]);
		roznica(liczbaA[1],liczbaB[1],pw[0],wynik[1], pw[1]);
		roznica(liczbaA[2],liczbaB[2],pw[1],wynik[2], pw[2]);
		roznica(liczbaA[3],liczbaB[3],pw[2],wynik[3], pw[3]);
		roznica(liczbaA[4],liczbaB[4],pw[3],wynik[4], pw[4]);
		roznica(liczbaA[5],liczbaB[5],pw[4],wynik[5], pw[5]);
		roznica(liczbaA[6],liczbaB[6],pw[5],wynik[6], pw[6]);
		roznica(liczbaA[7],liczbaB[7],pw[6],wynik[7], pw[7]);
		
		zw = wynik[7];
		ov = (~z1 & z2 & zw) | (z1 & ~z2 & zw);
	end
	
	else if(wybor == 2'b10)
	begin
		wynik = liczbaA | liczbaB;
	end
	
	else if(wybor == 2'b11)
	begin
		wynik = liczbaA & liczbaB;
	end
end
	 

assign C = pw[7];
assign EVEN = ^wynik;
assign Z = (wynik == 0);
assign OV = ov;

endmodule

\end{lstlisting}

\section{Przypisanie przycisków}

\begin{lstlisting}

NET "bitP"  LOC = "p94"  ;
NET "C"  LOC = "p104"  ;
NET "EVEN"  LOC = "p102"  ;
NET "liczbaA<0>"  LOC = "p2"  ;
NET "liczbaA<1>"  LOC = "p4"  ;
NET "liczbaA<2>"  LOC = "p6"  ;
NET "liczbaA<3>"  LOC = "p9"  ;
NET "liczbaA<4>"  LOC = "p3"  ;
NET "liczbaA<5>"  LOC = "p5"  ;
NET "liczbaA<6>"  LOC = "p7"  ;
NET "liczbaA<7>"  LOC = "p10"  ;
NET "liczbaB<0>"  LOC = "p133"  ;
NET "liczbaB<1>"  LOC = "p135"  ;
NET "liczbaB<2>"  LOC = "p138"  ;
NET "liczbaB<3>"  LOC = "p140"  ;
NET "liczbaB<4>"  LOC = "p134"  ;
NET "liczbaB<5>"  LOC = "p136"  ;
NET "liczbaB<6>"  LOC = "p139"  ;
NET "liczbaB<7>"  LOC = "p142"  ;
NET "OV"  LOC = "p97"  ;
NET "wybor<0>"  LOC = "p124"  ;
NET "wybor<1>"  LOC = "p39"  ;
NET "wynik<0>"  LOC = "p112"  ;
NET "wynik<1>"  LOC = "p114"  ;
NET "wynik<2>"  LOC = "p116"  ;
NET "wynik<3>"  LOC = "p118"  ;
NET "wynik<4>"  LOC = "p113"  ;
NET "wynik<5>"  LOC = "p115"  ;
NET "wynik<6>"  LOC = "p117"  ;
NET "wynik<7>"  LOC = "p119"  ;
NET "Z"  LOC = "p100"  ;

\end{lstlisting}