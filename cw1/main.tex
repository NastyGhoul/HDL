\section*{Treść zadania}
\begin{enumerate}
    \item Zaprojektować licznik rewersyjny liczący w przód od 0..9 i w tył od 9..0 w kodzie (UZUPEŁNIJ). Licznik powinien zawierać:
    \begin{itemize}
        \item COUNTING DIRECTION (CD). Wartość logiczna 0 oznacza liczenie w przód.
        \item SET VALUE (SV). Sygnał nadrzędny asynchroniczny, który ustawia aktywne zero wartość licznika.
        \item CLOCK ENABLE (CE). Licznik liczy przy CE $ = 1$ i nie liczy przy CE $= 0$.
    \end{itemize}
    \item Przeprowadzić symulację układu, w której w formie sekwencyjnej pojawiają się zmiany sygnałów: \\
    Zegarowego, CE, CD, SV. Sprawdzić wszystkie funkcje licznika.
\end{enumerate}

\begin{figure}[!htb]
    \centering
    \includegraphics[height=10cm]{./images/Counter.jpg}
\end{figure}

\clearpage

\section{Kod Verilog}

\begin{lstlisting}
	`timescale 1ns / 1ps

	module main(
		input [7:0] slowo_trans,
		input start_trans, input CLK,
		 input czy_parz, input jaki_parz,
		 output reg transmisja, output wyjscie_trans
		);
		 
	reg CLK2380;
	reg wyjscie;
	reg [3:0] stan;
	reg [10:0] zatrzasniete;
	reg [5:0] stan_licznika;
	
	initial
	begin
		stan <= 0;
		stan_licznika <= 0;
	end
	
	always @(posedge CLK)
	begin
		stan_licznika <= stan_licznika + 1;
		begin
	
		//if (stan_licznika == 6'b000001) //1
		if (stan_licznika == 6'b101010) //42
		begin
			stan_licznika <= 0;
			CLK2380 = 1'b1;
		end
		else
			CLK2380 = 1'b0;
		end
	end
	
	always @(slowo_trans or start_trans)
	begin
		if(start_trans == 0)
		begin
			zatrzasniete[0] <= 0;
			zatrzasniete[8:1] <= slowo_trans[7:0];
			
			if (jaki_parz == 0)
				zatrzasniete[9] <= (^slowo_trans[7:0]);
			else
				zatrzasniete[9] <= (~^slowo_trans[7:0]);
			
			zatrzasniete[10] <= 1;
		end
	end
	
	always @(posedge CLK2380 or negedge start_trans)
	begin
		if(start_trans == 0)
			stan <= 0;
		else if (stan == 4'b1001 && czy_parz == 0)
			stan <= stan + 2;
		else if (stan < 12)
			stan <= stan + 1;
	end
	
	always @(stan or zatrzasniete)
	begin
		case (stan)
			4'b0000 : wyjscie = 1;
			4'b0001 : wyjscie = zatrzasniete[0];
			4'b0010 : wyjscie = zatrzasniete[1];
			4'b0011 : wyjscie = zatrzasniete[2];
			4'b0100 : wyjscie = zatrzasniete[3];
			4'b0101 : wyjscie = zatrzasniete[4];
			4'b0110 : wyjscie = zatrzasniete[5];
			4'b0111 : wyjscie = zatrzasniete[6];
			4'b1000 : wyjscie = zatrzasniete[7];
			4'b1001 : wyjscie = zatrzasniete[8];
			4'b1010 : wyjscie = zatrzasniete[9];
			default : wyjscie = zatrzasniete[10];
		endcase
	end
	
	always @(stan)
	begin
			if(stan>0 && stan <12)
				transmisja = 1;
			else
				transmisja = 0;
	end
	assign wyjscie_trans = wyjscie;
	endmodule
\end{lstlisting}

\section{Kod w symulacji}

\begin{lstlisting}

	`timescale 1ns / 1ps

	module Sym1;
	
		// Inputs
		reg CLK;
		reg CE;
		reg CD;
		reg SV;
	
		// Outputs
		wire [3:0] out;
	
		// Instantiate the Unit Under Test (UUT)
		main uut (
			.CLK(CLK), 
			.CE(CE), 
			.CD(CD), 
			.SV(SV), 
			.out(out)
		);
		
		integer i;
		initial begin
			// Initialize Inputs
			CLK = 0;
			CE = 0;
			CD = 0;
			SV = 1;
	
			// Wait 100 ns for global reset to finish
			#50;
			
			//fork
			//begin
				CE = 1;
				for(i=0; i<11; i=i+1)
				begin
					#25 CLK = 1;
					#25 CLK = 0;
				end
			//end
				CE = 0;
				for(i=0; i<2; i=i+1)
				begin
					#25 CLK = 1;
					#25 CLK = 0;
				end
			
			#50;
			SV = 0;
			#50;
			SV = 1;
			//fork
			//begin
				CE = 1;
				for(i=0; i<11; i=i+1)
				begin
					#25 CLK = 1;
					#25 CLK = 0;
				end
			//end
				CE = 0;
				for(i=0; i<2; i=i+1)
				begin
					#25 CLK = 1;
					#25 CLK = 0;
				end
			
			CD = 1;
			
			#50;
			SV = 0;
			#50;
			SV = 1;
			
			CE = 1;
				for(i=0; i<11; i=i+1)
				begin
					#25 CLK = 1;
					#25 CLK = 0;
				end
			//end
				CE = 0;
				for(i=0; i<2; i=i+1)
				begin
					#25 CLK = 1;
					#25 CLK = 0;
				end
			
			#50;
			SV = 0;
			#50;
			SV = 1;
			//fork
			//begin
				CE = 1;
				for(i=0; i<11; i=i+1)
				begin
					#25 CLK = 1;
					#25 CLK = 0;
				end
			//end
				CE = 0;
				for(i=0; i<2; i=i+1)
				begin
					#25 CLK = 1;
					#25 CLK = 0;
				end
				
		end
	endmodule
	
	

\end{lstlisting}