\tableofcontents\clearpage

\section{Wstęp}
\subsection{Cel ćwiczenia}
\hspace*{0.5cm}Celem ćwiczenia było stworzenie układu realizującego funkcje licznika, wyświetlającego cyfry na wyświetlaczu siedmio segmentowym. Układ ma posiadać funkcję dwóch przycisków: Reset i Stop, a ich sposób implementacji jest dowolny. \\ \hspace*{0.5cm}Układ również powinien być zaprojektowany segmentowo, czyli każda funkcjonalność powinna być osobnym elementem. Napisane oczywiście w języku VHDL.

\vspace*{0.5cm}

\hspace*{0.5cm}W późniejszych pod-działach poszczególnych elementów ,,Działanie'' zostanie krótko opisana ich funkcjonalność. 

\subsection{Graficzne przedstawienie schematu}
\begin{figure}[!htb]
    \centering
    \includegraphics[width=18.4cm]{./Kod/schemat.png}
    \caption*{schemat układu VHDL}
\end{figure}

\subsection{Połączenie elementów przy pomocy kodu}
\input{Kod/schemat.tex}

\section{Elementy układu}
\subsection{Dzielnik częstotliwości na 1Hz}
\subsubsection{Działanie}
\hspace*{0.5cm}Sygnałem wejściowym i wyjściowym jest odpowiednio sygnał CLK i CLK 1Hz w formacie std logic. Zmienną pomocniczą jest ,,liczenie'' zdefiniowaną w zagresie do 14 bitów danych.

\vspace*{0.5cm}

\hspace*{0.5cm}Funkcjonalność dzielnika sprowadza się do czekania, aż główny sygnał zegarowy przeliczy do wybranej wartości. W przypadku dzielnika na częstotliwość 1Hz i częstotliwości zegara 100kHz, dzielnik czeka dokładnie 100k zdarzeń po czym zmienia wartość logiczna z $0$ na $1$ i resetuje się.
\subsubsection{Kod}
\input{Kod/czestotliwosc_1hz.tex}

\clearpage\subsection{Dzielnik częstotliwości na 400Hz}
\subsubsection{Działanie}
\hspace*{0.5cm}Dzielnik częstotliwości 400Hz działa dokładnie tak jak poprzedni dzielnik 1Hz z tą różnicą, że wartość przez którą jest dzielona częstotliwość zegara wynosi 250 i zakres sygnału ,,liczenie'' jest zdefiniowany do 8bitów.
\subsubsection{Kod}
\input{Kod/czestotliwosc_400hz.tex}

\subsection{Licznik}
\subsubsection{Działanie}
\hspace*{0.5cm}Sygnałami wejściowymi są: zegar, reset, stop w formacie std logic. Wyjściowymi sygnałami są kolejno: wyjscie1, wyjscie2, wyjscie3, wyjscie4 w formacie std logic vector (3 downto 0). Pomocniczymi sygnałami są: liczba1, liczba2, liczba3, liczba4 również w formacie std logic vector (3 downto 0), zdefiniowane początkowo jako zero.

\vspace*{0.2cm}

\hspace*{0.5cm}Co zmiane zegara na wartość $1$, zmienia się wartość zegara nadrzędnego również o $1$. Po osiągnięciu przez zegar wartości 9 (binarnie: 1001) w następnym cyklu jest zerowany i dodawana jest wartość $1$ do zegara podrzędnego. Na bierząco wartość zmiennej pomocniczej przekazywana jest do wyjścia.

\vspace*{0.5cm}

\hspace*{0.5cm}Zdecydowaliśmy się na implementacje licznika w postaci jednego modułu zamiast 4 osobnych, ze względu na prostotę i szybkość implementacji takiego rozwiązania (oczywiście kosztem przejrzystości).
\subsubsection{Kod}
\begin{lstlisting}
	always @(posedge CLK)
	begin
		stan_licznika <= stan_licznika + 1;
		begin
	
		//if (stan_licznika == 6'b000001) //1
		if (stan_licznika == 6'b101010) //42
		begin
			stan_licznika <= 0;
			CLK2380 = 1'b1;
		end
		else
			CLK2380 = 1'b0;
		end
	end
\end{lstlisting}

\subsection{Licznik modulo 4}
\subsubsection{Działanie}
\hspace*{0.5cm}Sygnałem wejściowym jest CLK 400Hz w formacie std logic, a wyjściowym kanal w formacie integer range 0 to 3. Sygnałem pomocniczym jest liczenie w formacie integer range 0 to 3.

\vspace*{0.5cm}

\hspace*{0.5cm}Licznik modulo 4 wybiera który z kanałów będzie rejestrowany i wyświetlany na wyświetlaczu siedmio segmentowym. Robi to w częstotliwości 400Hz. 
\subsubsection{Kod}
\begin{lstlisting}
    library IEEE;
    use IEEE.STD_LOGIC_1164.ALL;
    entity licznik4 is
        Port ( CLK_400Hz : in  STD_LOGIC;
               kanal : out integer range 0 to 3);
    end licznik4;
    
    architecture Behavioral of licznik4 is
    
    signal liczenie : integer range 0 to 3 := 0;
    begin
    process(CLK_400Hz)
    begin
        if CLK_400Hz'event and CLK_400Hz = '1' then
            if liczenie = 3 then
                liczenie <= 0;
            else
                liczenie <= liczenie + 1;
            end if;
        end if;
    end process;
    
    kanal <= liczenie;
    
    end Behavioral;
\end{lstlisting}

\subsection{Multiplekser}
\subsubsection{Działanie}
\hspace*{0.5cm}Sygnałami wejściowymi są: liczba1, liczba2, liczba3, liczba4 w formacie std logic vector (3 downto 0) i kanal w formacie std logic vector (1 downto 0). Wyjściowym sygnałem jest wyjscie w formacie std logic vector (3 downto 0).

\vspace*{0.5cm}

\hspace*{0.5cm}W zależności od wybranego kanału multiplekser przekazuje odpowiadającą mu liczbę do transkodera.
\subsubsection{Kod}
\begin{lstlisting}
    library IEEE;
    use IEEE.STD_LOGIC_1164.ALL;
    
    entity multiplekser is
        Port ( liczba1 : in  STD_LOGIC_VECTOR (3 downto 0);
               liczba2 : in  STD_LOGIC_VECTOR (3 downto 0);
               liczba3 : in  STD_LOGIC_VECTOR (3 downto 0);
               liczba4 : in  STD_LOGIC_VECTOR (3 downto 0);
               kanal : in	STD_LOGIC_VECTOR (1 downto 0);
               wyjscie : out  STD_LOGIC_VECTOR (3 downto 0));
    end multiplekser;
    
    architecture Behavioral of multiplekser is
    
    begin
    
        with kanal select
            wyjscie <= liczba1 when "00",
                        liczba2 when "01",
                        liczba3 when "10",
                        liczba4 when others;
    
    end Behavioral;
\end{lstlisting}

\subsection{Demultiplekser}
\subsubsection{Działanie}
\hspace*{0.5cm}Sygnałem wejściowym jest kanal w formacie std logic vector (1 downto 0), a wyjściowym anody w formacie std logic vector (3 downto 0).

\vspace*{0.5cm}

\hspace*{0.5cm}W zależności od wybranego kanału, aktywowane są odpowiadające im binarnie anody.
\subsubsection{Kod}
\input{Kod/demultiplekser.tex}

\subsection{Transkoder}
\subsubsection{Działanie}
\hspace*{0.5cm}Sygnałem wejściowym jest liczba w formacie std logic vector (3 downto 0), a wyjściowym liczba 7seg w formacie std logic vector (6 downto 0).

\vspace*{0.5cm}

\hspace*{0.5cm}W zależności od podanej liczby binarnej wysyła $0$ do odpowiednich katod by wyświetlić liczbę w systemie dziesiętnym.
\subsubsection{Kod}
\input{Kod/transkoder.tex}