\begin{lstlisting}
	`timescale 1ns / 1ps

	module main(
		input [7:0] slowo_trans,
		input start_trans, input CLK,
		 input czy_parz, input jaki_parz,
		 output reg transmisja, output wyjscie_trans
		);
		 
	reg CLK2380;
	reg wyjscie;
	reg [3:0] stan;
	reg [10:0] zatrzasniete;
	reg [5:0] stan_licznika;
	
	initial
	begin
		stan <= 0;
		stan_licznika <= 0;
	end
	
	always @(posedge CLK)
	begin
		stan_licznika <= stan_licznika + 1;
		begin
	
		//if (stan_licznika == 6'b000001) //1
		if (stan_licznika == 6'b101010) //42
		begin
			stan_licznika <= 0;
			CLK2380 = 1'b1;
		end
		else
			CLK2380 = 1'b0;
		end
	end
	
	always @(slowo_trans or start_trans)
	begin
		if(start_trans == 0)
		begin
			zatrzasniete[0] <= 0;
			zatrzasniete[8:1] <= slowo_trans[7:0];
			
			if (jaki_parz == 0)
				zatrzasniete[9] <= (^slowo_trans[7:0]);
			else
				zatrzasniete[9] <= (~^slowo_trans[7:0]);
			
			zatrzasniete[10] <= 1;
		end
	end
	
	always @(posedge CLK2380 or negedge start_trans)
	begin
		if(start_trans == 0)
			stan <= 0;
		else if (stan == 4'b1001 && czy_parz == 0)
			stan <= stan + 2;
		else if (stan < 12)
			stan <= stan + 1;
	end
	
	always @(stan or zatrzasniete)
	begin
		case (stan)
			4'b0000 : wyjscie = 1;
			4'b0001 : wyjscie = zatrzasniete[0];
			4'b0010 : wyjscie = zatrzasniete[1];
			4'b0011 : wyjscie = zatrzasniete[2];
			4'b0100 : wyjscie = zatrzasniete[3];
			4'b0101 : wyjscie = zatrzasniete[4];
			4'b0110 : wyjscie = zatrzasniete[5];
			4'b0111 : wyjscie = zatrzasniete[6];
			4'b1000 : wyjscie = zatrzasniete[7];
			4'b1001 : wyjscie = zatrzasniete[8];
			4'b1010 : wyjscie = zatrzasniete[9];
			default : wyjscie = zatrzasniete[10];
		endcase
	end
	
	always @(stan)
	begin
			if(stan>0 && stan <12)
				transmisja = 1;
			else
				transmisja = 0;
	end
	assign wyjscie_trans = wyjscie;
	endmodule
\end{lstlisting}